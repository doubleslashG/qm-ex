\documentclass{article}

\usepackage{amsmath}

\usepackage{bm}

\usepackage{physics}

\usepackage{listings}
\usepackage{color}

\usepackage{graphicx}
\usepackage{epstopdf}
\graphicspath{{img/}}

\begin{document}

\title{Quantum physics: II set of exercises - Solutions}
\author{Giancarlo Fissore}
\date{January 2015}
\maketitle

\section{Scattering theory}

\section{Variational methods}

\section{Perturbative methods}

The hamiltonian of the problem 

\begin{equation}
H = \frac{p^2}{2m} + \frac{1}{2} m \omega^2 x^2 + \lambda x^4
\end{equation}

is the hamiltonian of the linear harmonic oscillator

\begin{equation}
H_0 = \frac{p^2}{2m} + \frac{1}{2} m \omega^2 x^2
\end{equation}

plus a perturbative term

\begin{equation}
H_1 = \lambda x^4
\end{equation}

Eigenstates and eigenvalues of \( H_0 \) are well known and we refer to them as \( \ket{n^0} \) and \( E_n^0 \), with

\begin{equation}
E_n^0 = \left(n + \frac{1}{2} \right) \hbar \omega
\end{equation}

Recalling the expression of \( \bm{x} \) in terms of the raising and lowering operators \( \bm{a^+} \) and \( \bm{a} \)

\begin{equation}
\label{eq:x_def}
\bm{x} = \sqrt{ \frac{\hbar}{2m \omega}} \left( \bm{a^+} + \bm{a} \right)
\end{equation}

the perturbative hamiltonian reads

\begin{equation}
H_1 = \left( \frac{\hbar}{2m \omega} \right)^2 \left( \bm{a^+} + \bm{a} \right)^4
\end{equation}

The first order correction to the energy eigenvalues is given by

\begin{equation}
\label{eq:energy_correction}
E_n^1 = \bra{n^0} H_1 \ket{n^0} = \lambda \left( \frac{\hbar}{2m \omega} \right)^2 \bra{n^0} \left( \bm{a^+} + \bm{a} \right)^4 \ket{n^0}
\end{equation}

We then need to evaluate

\begin{align}
\label{eq:4expansion}
\left( \bm{a^+} + \bm{a} \right)^4 = & \left( {\bm{a^+}}^2 + \bm{a^+} \bm{a} + \bm{a} \bm{a^+} + \bm{a}^2 \right)^2 \nonumber \\
= & {\bm{a^+}}^4 + {\bm{a^+}}^3 \bm{a} + {\bm{a^+}}^2 \bm{a} \bm{a^+} + {\bm{a^+}}^2 \bm{a}^2 + \bm{a^+} \bm{a} {\bm{a^+}}^2 + (\bm{a^+}\bm{a})^2 + \nonumber \\ 
& \bm{a^+}\bm{a}^2\bm{a^+} + \bm{a^+} \bm{a}^3 + \bm{a} {\bm{a^+}}^3 + \bm{a} {\bm{a^+}}^2 \bm{a} + (\bm{a} \bm{a^+})^2 + \bm{a} \bm{a^+} \bm{a}^2 + \nonumber \\
& \bm{a}^2 {\bm{a^+}}^2 + \bm{a}^2 \bm{a^+} \bm{a} + \bm{a}^3 \bm{a^+} + \bm{a}^4
\end{align}

Recalling the action of \( \bm{a^+} \) and \( \bm{a} \) on the eigenstates \( \ket{n^0} \)

\begin{equation}
\bm{a^+} \ket{n^0} = \sqrt{n + 1} \ket{n^0 + 1}
\end{equation}

\begin{equation}
\bm{a} \ket{n^0} = \sqrt{n} \ket{n^0 - 1}
\end{equation}

and being the eigenstates of \( H_0 \) of course orthogonal

\begin{equation}
\braket{m^0}{n^0} = \delta_{mn}
\end{equation}

if we plug eq. \eqref{eq:4expansion} into eq. \eqref{eq:energy_correction} it is clear that the only terms which are not null are those in which \( \bm{a^+} \) and \(\bm{a} \) are applied the same number of times. Then

\begin{equation}
\label{eq:a_sandwich}
\bra{n^0} \left( \bm{a^+} + \bm{a} \right)^4 \ket{n^0}
\end{equation}

simplifies to

\begin{equation}
\label{eq:simplified_a_sandwich}
\bra{n^0} \left( {\bm{a^+}}^2 \bm{a}^2 + (\bm{a^+} \bm{a})^2 + \bm{a^+} \bm{a}^2 \bm{a^+} + \bm{a} {\bm{a^+}}^2 \bm{a} + (\bm{a} \bm{a^+})^2 + \bm{a}^2 {\bm{a^+}}^2 \right) \ket{n^0}
\end{equation}

Recalling the number operator \( \bm{ \hat{n} } \)

\begin{equation}
\bm{ \hat{n} } = \bm{a^+} \bm{a}, \quad \bm{ \hat{n} } \ket{n^0} = n\ket{n^0}
\end{equation}

and the relation

\begin{equation}
\bm{a} \bm{a^+} = \bm{a^+} \bm{a} + 1 = \bm{ \hat{n} } + 1
\end{equation}

we obtain

\begin{equation}
{\bm{a^+}}^2 \bm{a}^2 = \bm{a^+} (\bm{a^+} \bm{a}) \bm{a} = \bm{a^+} ( \bm{a} \bm{a^+} - 1) \bm{a} = (\bm{a^+} \bm{a})^2 - \bm{a^+} \bm{a} = \bm{ \hat{n} }^2 - \bm{ \hat{n} }
\end{equation}

\begin{equation}
(\bm{a^+} \bm{a})^2 = \bm{ \hat{n} }^2
\end{equation}

\begin{equation}
\bm{a^+} \bm{a}^2 \bm{a^+} = \bm{a^+} \bm{a} \bm{a} \bm{a^+} = \bm{a^+} \bm{a} (\bm{a^+} \bm{a} + 1) = \bm{ \hat{n} }^2 + \bm{ \hat{n} }
\end{equation}

\begin{equation}
\bm{a} {\bm{a^+}}^2 \bm{a} = (\bm{a} \bm{a^+})(\bm{a^+} \bm{a}) = (\bm{a^+} \bm{a} + 1) (\bm{a^+} \bm{a}) = \bm{ \hat{n} }^2 + \bm{ \hat{n} }
\end{equation}

\begin{equation}
(\bm{a} \bm{a^+})^2 = (\bm{a^+} \bm{a} + 1)^2 = \bm{ \hat{n} }^2 + 2 \bm{ \hat{n} } + 1
\end{equation}

\begin{align}
\bm{a}^2 {\bm{a^+}}^2 & = \bm{a} (\bm{a} \bm{a^+}) \bm{a^+} \nonumber \\
& = \bm{a} \bm{a^+} + (\bm{a} \bm{a^+})^2 \nonumber \\
& = 1 + \bm{a^+} \bm{a} + (1 + \bm{a^+} \bm{a})^2 \nonumber \\
& = \bm{ \hat{n} }^2 + 3 \bm{ \hat{n} } + 2
\end{align}

Eq. \eqref{eq:a_sandwich} then becomes


\begin{align}
\bra{n^0} \left( \bm{a^+} + \bm{a} \right)^4 \ket{n^0} & = \bra{n^0} \left( \bm{ 6 \hat{n} }^2 + 6 \bm{ \hat{n} } + 3 \right)^4 \ket{n^0} \nonumber \\
& = 3 (2n^2 + 2n +1) \braket{n^0}{n^0} \nonumber \\
& = 3(2n^2 + 2n +1)
\end{align}

and the first order correction to the energy \eqref{eq:energy_correction} reads

\begin{align}
\label{eq:x4correction}
E_n^1 & = \bra{n^0} H_1 \ket{n^0} \nonumber \\
& = \lambda \left( \frac{\hbar}{2m \omega} \right)^2 \bra{n^0} \left( \bm{a^+} + \bm{a} \right)^4 \ket{n^0} \nonumber \\
& = 3 \lambda \left( \frac{\hbar}{2m \omega} \right)^2 (2n^2 + 2n +1)
\end{align}

Energies for the total hamiltonian up to first order are then given by

\begin{align}
E_n & = E_n^0 + E_n^1 \nonumber \\
& = \left( n + \frac{1}{2} \right) \hbar \omega + 3 \lambda \left( \frac{\hbar^2}{2m\omega} \right)^2 (2n^2 + 2n +1)
\end{align}

For ground and first-excited states we have

\begin{equation}
E_0 = \frac{1}{2} \hbar \omega + 3 \lambda \left( \frac{\hbar}{2m\omega} \right)^2
\end{equation}

\begin{equation}
E_1 = \frac{3}{2} \hbar \omega + 15 \lambda \left( \frac{\hbar}{2m\omega} \right)^2
\end{equation}

\section{Numerical method}

\subsection{Dimensionless Schroedinger equation}
To make a significative comparison between results obtained by numerical solution and results obtained with other methods, it is necessary to numerically solve the dimensionless time-independent Schroedinger equation.

The hamiltonian of the problem is

\begin{equation}
H = \frac{p^2}{2m} + \lambda x^4
\end{equation}

and the explicit time-independent Schroedinger equation is

\begin{equation}
-\frac{\hbar^2}{2m} \frac{d^2}{dx^2} \psi(x) + \lambda x^4 \psi(x) = E \psi(x)
\end{equation}

By introducing the dimensionless variable \( y = \frac{x}{u} \) (where \( u \) is a constant to be determined) the operator of differentiation becomes

\begin{equation}
\frac{d}{dx} \psi \left( \frac{x}{u} \right) = \frac{1}{u} \frac{d}{dy} \psi(y) \quad \Rightarrow \quad \frac{d}{dx} = \frac{1}{u} \frac{d}{dy}
\end{equation}

and the Schroedinger equation becomes

\begin{equation}
-\frac{\hbar^2}{2m} \frac{1}{u^2} \frac{d^2}{dy^2} \psi(y) + \lambda u^4 y^4 \psi(y) = E \psi(y)
\end{equation}

With a little manipulation we obtain

\begin{equation}
\left( -\frac{1}{2}\frac{d^2}{dy^2} + \frac{m\lambda}{\hbar^2} u^6 y^4 \right) \psi(y) = \frac{mu^2}{\hbar^2} E \psi(y)
\end{equation}

For the left hand side of last equation to be completely dimensionless we must have

\begin{equation}
\frac{m\lambda}{\hbar^2} u^6 = 1 \quad \Rightarrow \quad u = \left( \frac{\hbar^2}{m\lambda} \right)^{\frac{1}{6}}
\end{equation}

The dimensionless equation to solve is then

\begin{equation}
\left( -\frac{1}{2}\frac{d^2}{dy^2} + y^4 \right) \psi(y) = \left(\frac{m}{\hbar^4 \lambda} \right)^{\frac{1}{3}} E \psi(y) = \varepsilon \psi(y)
\end{equation}

\subsection{Numerical solution}

To solve the problem numerically, it is necessary to discretize the variable \( y \) and express the hamiltonian of the problem in matrix form. Then diagonalizing \( H \) the eigenvalues are easily obtained.

Using \( N \) steps spaced by a distance \( d \), the following notation is introduced

\[ \psi(y_n) = \psi_n, \qquad \psi(y_n + d) = \psi_{n+1} \]

The one-dimensional laplacian is then easily discretized as follows

\begin{align*}
f'(x) & \simeq \frac{f\left(x+\frac{d}{2}\right) - f\left(x-\frac{d}{2}\right)}{d} \\ \\
f''(x) & \simeq \frac{f'\left(x+\frac{d}{2}\right) - f'\left(x-\frac{d}{2}\right)}{d} \\
& \simeq \frac{f\left(x+d\right) - f(x) - f(x) + f\left(x-d\right)}{d^2} \\
& = \frac{f(x+d) - 2f(x) + f(x-d)}{d^2}
\end{align*}

\begin{equation}
\left. \frac{d^2\psi}{dy^2} \right\vert_n = \frac{\psi_{n+1} - 2\psi_n + \psi_{n-1}}{d^2}
\end{equation}

and the discretized Schroedinger equation is

\begin{equation}
\label{eq:discrete_se}
-\frac{1}{2d^2} \left( \psi_{n+1} - 2\psi_n + \psi_{n-1} \right) + V_n \psi_n = \varepsilon \psi_n, \quad V_n = y_n^4
\end{equation}

Eq. \eqref{eq:discrete_se} defines a system of equations. Setting the boundary conditions \( \psi_{-1} = \psi_{N+1} = 0 \) and introducing the vertical vector \( \psi = (\psi_0, \psi_1, \dots, \psi_N) \) the system can be expressed in matrix form as

\begin{equation}
-\frac{1}{2d^2}
  \begin{pmatrix}
    -2 & 1 &&& \\
    1 & -2 & 1 && \\
    & \ddots & \ddots & \ddots & \\
    && 1 & -2 & 1 \\
    &&& 1 & -2
  \end{pmatrix}
\psi +
  \begin{pmatrix}
    V_0 \\
    & V_1 \\
    && \ddots \\
    &&& V_{N-1} \\
    &&&& V_N
  \end{pmatrix}
\psi = \varepsilon \psi
\end{equation}

The boundary conditions have been set to obtain a square matrix which is easily diagonalizable and they force us to solve the problem in a region large enough for the eigenstates to be considered null at the boundaries. As the potential is rapidly increasing for \( |y| > 1 \) this is not a critical issue in our problem.

The hamiltonian in matrix form is the following

\begin{equation}
H = -\frac{1}{2d^2}
  \begin{pmatrix}
    -2+V_0 & 1 &&& \\
    1 & -2+V_1 & 1 && \\
    & \ddots & \ddots & \ddots & \\
    && 1 & -2-V_{N-1} & 1 \\
    &&& 1 & -2+V_N
  \end{pmatrix}
\end{equation}

And the eigenvalues problem is

\begin{equation}
H\psi = \varepsilon \psi
\end{equation}

Using Octave, the following code defines the problem and solves for the eigenvalues

\lstinputlisting[language=Octave]{octave/eigFromV.m}

The obtained result is

\begin{equation}
\varepsilon_0 = 0.66798
\end{equation}

\subsection{Solution by perturbation theory}
The hamiltonian of the problem is that of a free particle plus a perturbative term \( H_1 = \lambda x^4 \). Directly applying perturbation theory with such perturbative term gives a first order correction to the energy which is divergent. In order to get a valid correction term, we can express our hamiltonian in the equivalent form

\begin{equation}
H = \frac{p^2}{2m} + \frac{1}{2}m\omega^2x^2 + \lambda x^4 - \frac{1}{2}m\omega^2 x^2
\end{equation}

where the unperturbed part is the hamiltonian of the harmonic oscillator

\begin{equation}
H_0 = \frac{p^2}{2m} + \frac{1}{2}m\omega^2x^2
\end{equation}

and the perturbative part is

\begin{equation}
H_1 = \lambda x^4 - \frac{1}{2}m\omega^2 x^2
\end{equation}

For the problem to be well defined we must be able to retrieve the original unperturbed hamiltonian (that of the free particle) for \( \lambda \rightarrow 0 \). Then we must have \( \omega = \omega(\lambda) \) such that \( \omega(\lambda) \rightarrow 0 \) for \( \lambda \rightarrow 0 \) (the explicit expression for \( \lambda \) will be derived later).

To solve the problem we proceed as in exercise 3. The first order correction term for the energy is given by

\begin{equation}
\bra{n^0} H_1 \ket{n^0} = \lambda \bra{n^0}\bm{x}^4 \ket{n^0} - \frac{1}{2}m\omega^2 \bra{n^0}\bm{x}^2\ket{n^0}
\end{equation}

where the first part is the same as in exercise 3 (eq. \eqref{eq:x4correction})

\begin{equation}
\lambda \bra{n^0}\bm{x}^4 \ket{n^0} = 3 \lambda \left( \frac{\hbar}{2m \omega} \right)^2 (2n^2 + 2n +1)
\end{equation}

while for the second term we have (recalling eq. \eqref{eq:x_def} and discarding non-diagonal terms)

\begin{align}
\frac{1}{2} m \omega^2 \bra{n^0} \bm{x}^2 \ket{n^0} & = \frac{\hbar\omega}{4} \bra{n^0} \left( \bm{a^+} + \bm{a} \right)^2 \ket{n^0} \nonumber \\
& = \frac{\hbar\omega}{4} \bra{n^0} \left( {\bm{a^+}}^2 + \bm{a^+}\bm{a} + \bm{a}\bm{{a}^+} + \bm{a}^2 \right) \ket{n^0} \nonumber \\
& = \frac{\hbar\omega}{4} \bra{n^0} \left(\bm{a^+}\bm{a} + \bm{a}\bm{{a}^+}\right) \ket{n^0} \nonumber \\
& = \frac{\hbar\omega}{4} \bra{n^0} \left(2\bm{\hat{n}} + 1\right) \ket{n^0} \nonumber \\
& = \frac{\hbar\omega}{4} \left(2n + 1\right) \braket{n^0}{n^0} \nonumber \\
& = \frac{\hbar\omega}{4} \left(2n + 1\right)
\end{align}

The energy of the ground state up to first order is then given by

\begin{align}
\label{eq:energy_corrected}
E_0 & = E_0^0 + E_0^{1,quartic} - E_0^{1,quadratic} \nonumber \\
& = \frac{1}{2} \hbar \omega + \frac{3}{4} \lambda \left(\frac{\hbar}{m\omega} \right)^2 - \frac{1}{4} \hbar \omega \nonumber \\
& = \frac{1}{4} \hbar \omega + \frac{3}{4} \lambda \left(\frac{\hbar}{m\omega} \right)^2
\end{align}

The use of the hamiltonian of the harmonic oscillator to solve the problem suggests to express the result in units of \( \hbar \omega \). We can then derive an explicit expression for \( \omega \) enforcing the following condition

\begin{equation}
\hbar \omega = \lambda \left( \frac{\hbar}{n \omega} \right)^2 \quad \Rightarrow \quad \omega = \left( \frac{\hbar \lambda}{m^2} \right)^{\frac{1}{3}}
\end{equation}

Substituting the expression for \( \omega \) into eq. \eqref{eq:energy_corrected} we obtain

\begin{equation}
E_0 = 1 \left( \frac{\hbar^4 \lambda}{m^2} \right)^{\frac{1}{3}}
\end{equation}

and the dimensionless value of energy is

\begin{equation}
\varepsilon_{0,pert} = 1
\end{equation}

\end{document}
