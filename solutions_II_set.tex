\documentclass{article}

\usepackage{amsmath}

\usepackage{bm}

\usepackage{physics}

\usepackage{graphicx}
\usepackage{epstopdf}
\graphicspath{{img/}}

\begin{document}

\title{Quantum physics: II set of exercises - Solutions}
\author{Giancarlo Fissore}
\date{January 2015}
\maketitle

\section{Scattering theory}

\section{Variational methods}

\section{Perturbative methods}

The hamiltonian of the problem 

\begin{equation}
H = \frac{p^2}{2m} + \frac{1}{2} m \omega^2 x^2 + \lambda x^4
\end{equation}

is the hamiltonian of the linear harmonic oscillator

\begin{equation}
H_0 = \frac{p^2}{2m} + \frac{1}{2} m \omega^2 x^2
\end{equation}

plus a perturbative term

\begin{equation}
H_1 = \lambda x^4
\end{equation}

Eigenstates and eigenvalues of \( H_0 \) are well known and we refer to them as \( \ket{n^0} \) and \( E_n^0 \), with

\begin{equation}
E_n^0 = \left(n + \frac{1}{2} \right) \hbar \omega
\end{equation}

Recalling the expression of \( \bm{x} \) in terms of the raising and lowering operators \( \bm{a^+} \) and \( \bm{a} \)

\begin{equation}
\bm{x} = \sqrt{ \frac{\hbar}{2m \omega}} \left( \bm{a^+} + \bm{a} \right)
\end{equation}

the perturbative hamiltonian reads

\begin{equation}
H_1 = \left( \frac{\hbar}{2m \omega} \right)^2 \left( \bm{a^+} + \bm{a} \right)^4
\end{equation}

The first order correction to the energy eigenvalues is given by

\begin{equation}
\label{eq:energy_correction}
E_n^1 = \bra{n^0} H_1 \ket{n^0} = \lambda \left( \frac{\hbar}{2m \omega} \right)^2 \bra{n^0} \left( \bm{a^+} + \bm{a} \right)^4 \ket{n^0}
\end{equation}

We then need to evaluate

\begin{align}
\label{eq:4expansion}
\left( \bm{a^+} + \bm{a} \right)^4 = & \left( {\bm{a^+}}^2 + \bm{a^+} \bm{a} + \bm{a} \bm{a^+} + \bm{a}^2 \right)^2 \nonumber \\
= & {\bm{a^+}}^4 + {\bm{a^+}}^3 \bm{a} + {\bm{a^+}}^2 \bm{a} \bm{a^+} + {\bm{a^+}}^2 \bm{a}^2 + \bm{a^+} \bm{a} {\bm{a^+}}^2 + (\bm{a^+}\bm{a})^2 + \nonumber \\ 
& \bm{a^+}\bm{a}^2\bm{a^+} + \bm{a^+} \bm{a}^3 + \bm{a} {\bm{a^+}}^3 + \bm{a} {\bm{a^+}}^2 \bm{a} + (\bm{a} \bm{a^+})^2 + \bm{a} \bm{a^+} \bm{a}^2 + \nonumber \\
& \bm{a}^2 {\bm{a^+}}^2 + \bm{a}^2 \bm{a^+} \bm{a} + \bm{a}^3 \bm{a^+} + \bm{a}^4
\end{align}

Recalling the action of \( \bm{a^+} \) and \( \bm{a} \) on the eigenstates \( \ket{n^0} \)

\begin{equation}
\bm{a^+} \ket{n^0} = \sqrt{n + 1} \ket{n^0 + 1}
\end{equation}

\begin{equation}
\bm{a} \ket{n^0} = \sqrt{n} \ket{n^0 - 1}
\end{equation}

and being the eigenstates of \( H_0 \) of course orthogonal

\begin{equation}
\braket{m^0}{n^0} = \delta_{mn}
\end{equation}

if we plug eq. \eqref{eq:4expansion} into eq. \eqref{eq:energy_correction} it is clear that the only terms which are not null are those in which \( \bm{a^+} \) and \(\bm{a} \) are applied the same number of times. Then

\begin{equation}
\label{eq:a_sandwich}
\bra{n^0} \left( \bm{a^+} + \bm{a} \right)^4 \ket{n^0}
\end{equation}

simplifies to

\begin{equation}
\label{eq:simplified_a_sandwich}
\bra{n^0} \left( {\bm{a^+}}^2 \bm{a}^2 + (\bm{a^+} \bm{a})^2 + \bm{a^+} \bm{a}^2 \bm{a^+} + \bm{a} {\bm{a^+}}^2 \bm{a} + (\bm{a} \bm{a^+})^2 + \bm{a}^2 {\bm{a^+}}^2 \right) \ket{n^0}
\end{equation}

Recalling the number operator \( \bm{ \hat{n} } \)

\begin{equation}
\bm{ \hat{n} } = \bm{a^+} \bm{a}, \quad \bm{ \hat{n} } \ket{n^0} = n\ket{n^0}
\end{equation}

and the relation

\begin{equation}
\bm{a} \bm{a^+} = \bm{a^+} \bm{a} + 1 = \bm{ \hat{n} } + 1
\end{equation}

we obtain

\begin{equation}
{\bm{a^+}}^2 \bm{a}^2 = \bm{a^+} (\bm{a^+} \bm{a}) \bm{a} = \bm{a^+} ( \bm{a} \bm{a^+} - 1) \bm{a} = (\bm{a^+} \bm{a})^2 - \bm{a^+} \bm{a} = \bm{ \hat{n} }^2 - \bm{ \hat{n} }
\end{equation}

\begin{equation}
(\bm{a^+} \bm{a})^2 = \bm{ \hat{n} }^2
\end{equation}

\begin{equation}
\bm{a^+} \bm{a}^2 \bm{a^+} = \bm{a^+} \bm{a} \bm{a} \bm{a^+} = \bm{a^+} \bm{a} (\bm{a^+} \bm{a} + 1) = \bm{ \hat{n} }^2 + \bm{ \hat{n} }
\end{equation}

\begin{equation}
\bm{a} {\bm{a^+}}^2 \bm{a} = (\bm{a} \bm{a^+})(\bm{a^+} \bm{a}) = (\bm{a^+} \bm{a} + 1) (\bm{a^+} \bm{a}) = \bm{ \hat{n} }^2 + \bm{ \hat{n} }
\end{equation}

\begin{equation}
(\bm{a} \bm{a^+})^2 = (\bm{a^+} \bm{a} + 1)^2 = \bm{ \hat{n} }^2 + 2 \bm{ \hat{n} } + 1
\end{equation}

\begin{align}
\bm{a}^2 {\bm{a^+}}^2 & = \bm{a} (\bm{a} \bm{a^+}) \bm{a^+} \nonumber \\
& = \bm{a} \bm{a^+} + (\bm{a} \bm{a^+})^2 \nonumber \\
& = 1 + \bm{a^+} \bm{a} + (1 + \bm{a^+} \bm{a})^2 \nonumber \\
& = \bm{ \hat{n} }^2 + 3 \bm{ \hat{n} } + 2
\end{align}

Eq. \eqref{eq:a_sandwich} then becomes


\begin{align}
\bra{n^0} \left( \bm{a^+} + \bm{a} \right)^4 \ket{n^0} & = \bra{n^0} \left( \bm{ 6 \hat{n} }^2 + 6 \bm{ \hat{n} } + 3 \right)^4 \ket{n^0} \nonumber \\
& = 3 (2n^2 + 2n +1) \braket{n^0}{n^0} \nonumber \\
& = 3(2n^2 + 2n +1)
\end{align}

and the first order correction to the energy \eqref{eq:energy_correction} reads

\begin{align}
E_n^1 & = \bra{n^0} H_1 \ket{n^0} \nonumber \\
& = \lambda \left( \frac{\hbar}{2m \omega} \right)^2 \bra{n^0} \left( \bm{a^+} + \bm{a} \right)^4 \ket{n^0} \nonumber \\
& = 3 \lambda \left( \frac{\hbar}{2m \omega} \right)^2 (2n^2 + 2n +1)
\end{align}

Energies for the total hamiltonian up to first order are then given by

\begin{align}
E_n & = E_n^0 + E_n^1 \nonumber \\
& = \left( n + \frac{1}{2} \right) \hbar \omega + 3 \lambda \left( \frac{\hbar^2}{2m\omega} \right)^2 (2n^2 + 2n +1)
\end{align}

For ground and first-excited states we have

\begin{equation}
E_0 = \frac{1}{2} \hbar \omega + 3 \lambda \left( \frac{\hbar}{2m\omega} \right)^2
\end{equation}

\begin{equation}
E_1 = \frac{3}{2} \hbar \omega + 15 \lambda \left( \frac{\hbar}{2m\omega} \right)^2
\end{equation}

\section{Numerical method}

\subsection{Dimensionless Schroedinger equation}
To make a significative comparison between results obtained by numerical solution and results obtained with other methods, it is necessary to numerically solve the dimensionless time-independent Schroedinger equation.

The hamiltonian of the problem is

\begin{equation}
H = \frac{p^2}{2m} + \lambda x^4
\end{equation}

and the explicit time-independent Schroedinger equation is

\begin{equation}
-\frac{\hbar^2}{2m} \frac{d^2}{dx^2} \psi(x) + \lambda x^4 \psi(x) = E \psi(x)
\end{equation}

By introducing the dimensionless variable \( y = \frac{x}{u} \) (where \( u \) is a constant to be determined) the operator of differentiation becomes

\begin{equation}
\frac{d}{dx} \psi \left( \frac{x}{u} \right) = \frac{1}{u} \frac{d}{dy} \psi(y) \quad \Rightarrow \quad \frac{d}{dx} = \frac{1}{u} \frac{d}{dy}
\end{equation}

and the Schroedinger equation becomes

\begin{equation}
-\frac{\hbar^2}{2m} \frac{1}{u^2} \frac{d^2}{dy^2} \psi(y) + \lambda u^4 y^4 \psi(y) = E \psi(y)
\end{equation}

With a little manipulation we obtain

\begin{equation}
\left( -\frac{1}{2}\frac{d^2}{dy^2} + \frac{m\lambda}{\hbar^2} u^6 y^4 \right) \psi(y) = \frac{mu^2}{\hbar^2} E \psi(y)
\end{equation}

For the left hand side of last equation to be completely dimensionless we must have

\begin{equation}
\frac{m\lambda}{\hbar^2} u^6 = 1 \quad \Rightarrow \quad u = \left( \frac{\hbar^2}{m\lambda} \right)^{\frac{1}{6}}
\end{equation}

The dimensionless equation to solve is then

\begin{equation}
\left( -\frac{1}{2}\frac{d^2}{dy^2} + y^4 \right) \psi(y) = \left(\frac{m}{\hbar^4 \lambda} \right) E \psi(y) = \varepsilon \psi(y)
\end{equation}

\end{document}
